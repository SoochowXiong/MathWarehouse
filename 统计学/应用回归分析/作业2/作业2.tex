\documentclass[12pt, a4paper, oneside]{ctexart}
\usepackage{amsmath, amsthm, amssymb, bm, color, framed, graphicx, hyperref, mathrsfs,enumerate}

\title{\textbf{第二次作业}}
\author{1907402030 熊雄}
\date{\today}
\linespread{1.15}
\definecolor{shadecolor}{RGB}{241, 241, 255}
\newcounter{problemname}
\newenvironment{problem}{\begin{shaded}\stepcounter{problemname}\par\noindent\textbf{题目\arabic{problemname}. }}{\end{shaded}\par}
\newenvironment{solution}{\par\noindent\textbf{解答. }}{\par}
\def\hth{\hat\theta}
\def\mA{\mathcal{A}}
\def\vv{1.2}
\usepackage{indentfirst}
\usepackage{xcolor} 
\usepackage{listings}
\lstset{%
	alsolanguage=Java,
	%alsolanguage=[ANSI]C,      %可以添加很多个alsolanguage,如alsolanguage=matlab,alsolanguage=VHDL等
	alsolanguage= matlab,
	alsolanguage= XML,
	alsolanguage= R,
	tabsize=4, %
	frame=shadowbox, %把代码用带有阴影的框圈起来
	commentstyle=\color{red!50!green!50!blue!50},%浅灰色的注释
	rulesepcolor=\color{red!20!green!20!blue!20},%代码块边框为淡青色
	keywordstyle=\color{blue!90}\bfseries, %代码关键字的颜色为蓝色,粗体
	showstringspaces=false,%不显示代码字符串中间的空格标记
	stringstyle=\ttfamily, % 代码字符串的特殊格式
	keepspaces=true, %
	breakindent=22pt, %
	numbers=left,%左侧显示行号 往左靠,还可以为right,或none,即不加行号
	stepnumber=1,%若设置为2,则显示行号为1,3,5,即stepnumber为公差,默认stepnumber=1
	%numberstyle=\tiny, %行号字体用小号
	numberstyle={\color[RGB]{0,192,192}\tiny} ,%设置行号的大小,大小有tiny,scriptsize,footnotesize,small,normalsize,large等
	numbersep=8pt,  %设置行号与代码的距离,默认是5pt
	basicstyle=\footnotesize, % 这句设置代码的大小
	showspaces=false, %
	flexiblecolumns=true, %
	breaklines=true, %对过长的代码自动换行
	breakautoindent=true,%
	breakindent=4em, %
	escapebegin=\begin{CJK*},escapeend=\end{CJK*},
	aboveskip=1em, %代码块边框
	tabsize=2,
	showstringspaces=false, %不显示字符串中的空格
	backgroundcolor=\color[RGB]{245,245,244},   %代码背景色
	%backgroundcolor=\color[rgb]{0.91,0.91,0.91}    %添加背景色
	escapeinside=``,  %在``里显示中文
	%% added by http://bbs.ctex.org/viewthread.php?tid=53451
	fontadjust,
	captionpos=t,
	framextopmargin=2pt,framexbottommargin=2pt,abovecaptionskip=-3pt,belowcaptionskip=3pt,
	xleftmargin=4em,xrightmargin=4em, % 设定listing左右的空白
	texcl=true,
	% 设定中文冲突,断行,列模式,数学环境输入,listing数字的样式
	extendedchars=false,columns=flexible,mathescape=true
	% numbersep=-1em
}

\begin{document}
 	\maketitle
\begin{problem}
	编写 R 代码来检验下面的两个等式.
	\begin{enumerate}
  \item	{\tt}  $ \mathbb{E} \lbrack \mathbb{E}(X \lvert Y)\rbrack = \mathbb{E} \lbrack X \rbrack $;
  \item	{\tt}   $ Var(X) = Var \lbrack \mathbb{E}(X \lvert Y)\rbrack + \mathbb{E} \lbrack Var(X \lvert Y) \rbrack $.
 	\end{enumerate}
\end{problem}

\begin{solution}
	主要思路:利用lec2.ppt的第8页的数据编写R代码验证这两个等式.
	\begin{enumerate}
	\item	{\tt} 编写R代码如下:
	\lstinputlisting[language=R]{1.1_code.r}
	运行后得到:$ \mathbb{E} \lbrack \mathbb{E}(X \lvert Y)\rbrack = \mathbb{E} \lbrack X \rbrack =\frac{1}{3}$, 等式1成立.

	\item	{\tt} 编写R代码如下:
	\lstinputlisting[language=R]{1.2_code.r}
	但是不知道为什么最后的结果不等于0. 调试了半天, 得到result始终为一个1*6的矩阵, 矩阵内的每个元素都为464.3386. 
	
	
\end{solution}

\begin{problem}
	编写R代码重新计算lec2.ppt第20页的式子.
\end{problem}

\begin{solution}
	编写R代码如下:
	\lstinputlisting[language=R]{2_code.r}
	运行后可以得到:
	b0 = 10.2779285495247, b1=4.91933072677094, 从而得到回归方程为:
	\[ \hat{y} = 10.280 + 4.919x ,\]
	验证成功.
\end{solution}
  
  
\end{document}
