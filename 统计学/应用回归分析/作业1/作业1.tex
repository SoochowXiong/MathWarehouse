\documentclass[12pt, a4paper, oneside]{ctexart}
\usepackage{amsmath, amsthm, amssymb, bm, color, framed, graphicx, hyperref, mathrsfs,enumerate}

\title{\textbf{第一次作业}}
\author{1907402030 熊雄}
\date{\today}
\linespread{1.15}
\definecolor{shadecolor}{RGB}{241, 241, 255}
\newcounter{problemname}
\newenvironment{problem}{\begin{shaded}\stepcounter{problemname}\par\noindent\textbf{题目\arabic{problemname}. }}{\end{shaded}\par}
\newenvironment{solution}{\par\noindent\textbf{解答. }}{\par}
\newenvironment{note}{\par\noindent\textbf{题目\arabic{problemname}的注记. }}{\par}
\def\hth{\hat\theta}
\def\mA{\mathcal{A}}
\def\vv{1.2}
\usepackage{indentfirst}

 \begin{document}
 	\maketitle
\begin{problem}
 	设$y_{1},\dots,y_{n}$是一组样本,其中$\mu$和$\sigma$都是未知的。构建下面模型
 	$$
 	y_{i}=\mu+\varepsilon_{i},~~i=1,2,\dots,n
 	$$
 	我们可以采用
 	\begin{itemize}
 		\item 最小二乘法估计: 最小化$\sum_{i=1}^{n}(y_{i}-\mu)^2$
 		\item 最小绝对值估计: 最小化$\sum_{i=1}^{n}|y_{i}-\mu|$
 	\end{itemize}
 	回答下面问题:
 	\begin{enumerate}
 		\item {\tt 证明$\mu$的最小二乘估计是样本均值;}
 		\item {\tt 证明$\mu$的最小绝对值估计是样本中位数;}
 		\item {\tt 列出样本均值的一个优点和一个缺点;}
 		\item {\tt 列出样本中位数的一个优点和一个缺点;}
 		\item {\tt 你会选择$\mu$的两个估计量的哪一个?说出理由.}
 	\end{enumerate}
\end{problem}



\begin{solution}
	
   	\begin{enumerate}
   	\item {\tt}    {\bf 证明:} 
   	设$f(t)=\sum_{i=1}^{n}(y_{i}-t)^2$, 对$t$求导得到:$$f'(t)=-2\sum_{i=1}^{n}(y_{i}-t)$$
   	令$f'(t)=0$, 可以得到
   	$$t_0=\frac{1}{n}\sum_{i=1}^{n}y_{i}$$
   	又因为$f''(t_0)=2n>0$, 故$t_0=\frac{1}{n}\sum_{i=1}^{n}y_{i}$为$f(\mu)$的极小值.而我们知道$\frac{1}{n}\sum_{i=1}^{n}y_{i}$就是样本均值$\mu$.故 $\mu$的最小二乘估计是样本均值,证毕.
   	
   	\item {\tt }   {\bf 证明:} 
   	设样本中位数为$y_{mid}$. 由于$y_{1},\dots,y_{n}$是一组样本的观测值,则将其任意排序后总能使其递增排列,故我们不妨假设其为递增数列,即
   	\[y_{1} \leq y_{2} \leq \dots \leq y_{n}\]
   		\begin{enumerate}
   	\item {\tt } 当$n$为奇数时,记$U_i=[y_i , y_{n+1-i}]$,$i=1,2,\dots, \frac{n+1}{2}$. 我们知道, $\left|y_1 - a \right|+\left|y_n - a \right| $在$ a\in U_1$时取得最小值 ; $\left|y_2 - a \right|+\left|y_{n-1} - a \right| $ 在$ a\in U_2$时取得最小值......以此类推, 最后只剩下$\left|y_{\frac{n+1}{2}} - a \right|$在$a\in U_{\frac{n+1}{2}}$时取得最小值, 即$a=y_{\frac{n+1}{2}}=y_{mid}$. 将所有的$U_i$取交集:
   	\[ \cap_{i=1}^{\frac{n+1}{2}}U_i=U_{\frac{n+1}{2}}=y_{mid} \]
   	即$a=y_{mid}$时使得所求离差绝对值的和最小, $\mu$的最小绝对值估计是样本中位数.
   	
   		\item {\tt } 当$n$为偶数时, 记$V_i=[y_i , y_{n+1-i}]$,$i=1,2,\dots, \frac{n}{2}$. 我们知道, $\left|y_1 - a \right|+\left|y_n - a \right| $在$ a\in V_1$时取得最小值 ; $\left|y_2 - a \right|+\left|y_{n-1} - a \right| $ 在$ a\in V_2$时取得最小值......以此类推, 最后只剩下$\left|y_{\frac{n}{2}} - a \right|+\left|y_{\frac{n}{2}+1} - a \right|$ 在 $a\in V_{\frac{n}{2}}$时取得最小值. 将所有的$V_i$取交集:
   		\[ \cap_{i=1}^{\frac{n}{2}}V_i=V_{\frac{n}{2}}=y_{mid} \]
   		即$a=y_{mid}$时使得所求离差绝对值的和最小, $\mu$的最小绝对值估计是样本中位数.
       \end{enumerate}
   综上所述,  $\mu$的最小绝对值估计是样本中位数, 证毕.
   	
   	
   	
   	\item {\tt }   {\bf 答:}
   	 \begin{enumerate}
   	 		\item {\tt }   {\bf 样本均值一个优点:}样本均值为总体均值的无偏估计、有效估计和相合估计,是总体均值的最优的估计量;
   	 		\item {\tt }   {\bf 样本均值一个缺点:}样本均值容易受极端值影响。当一组观测值为明显的偏态分布时,用样本均值来估计总体均值的效果比较差。
   	 \end{enumerate}
   	\item {\tt }   {\bf 答:}
   		 \begin{enumerate}
   		\item {\tt }   {\bf 样本中位数一个优点:}样本中位数不受极端值影响;
   		\item {\tt }   {\bf 样本中位数一个缺点:}当样本容量较大时,计算样本中位数比较复杂,比如若采用快速排序算法的时间复杂度为$ O(n\log(n)) $。
   		  \end{enumerate}
   	\item {\tt }   {\bf 答:}我会选择样本均值. 由第3问的样本均值的优点可知,样本均值是总体均值的最优的估计量. 我们研究样本的观测值的主要目的就是来估计总体, 而样本中位数不能很好地反映总体的特征,因此我会选择样本均值.
   \end{enumerate}
    
    
\end{solution}



  
  
  
\end{document}
