\documentclass[12pt, a4paper, oneside]{ctexart}
\usepackage{amsmath, amsthm, amssymb, bm, color, framed, graphicx, hyperref, mathrsfs,enumerate,colortbl}
\title{\textbf{第六次作业}}
\author{1907402030 熊雄}
\date{\today}
\linespread{1.25}
\definecolor{shadecolor}{RGB}{241, 241, 255}
\newcounter{problemname}
\newenvironment{problem}{\begin{shaded}\stepcounter{problemname}\par\noindent  \textbf{Question \arabic{problemname}. }} {\end{shaded}\par}
\newenvironment{solution}{\par\noindent \textbf {Answer. }}{\par}
\usepackage{indentfirst}
\usepackage{booktabs}
\usepackage{graphicx}
\usepackage{xcolor} 
\usepackage{listings}
\lstset{%
	alsolanguage=Java,
	%alsolanguage=[ANSI]C,      %可以添加很多个alsolanguage,如alsolanguage=matlab,alsolanguage=VHDL等
	alsolanguage= matlab,
	alsolanguage= XML,
	alsolanguage= R,
	tabsize=4, %
	frame=shadowbox, %把代码用带有阴影的框圈起来
	commentstyle=\color{red!100!green!100!blue!50},%浅灰色的注释
	rulesepcolor=\color{red!20!green!20!blue!20},%代码块边框为淡青色
	keywordstyle=\color{blue!90}\bfseries, %代码关键字的颜色为蓝色,粗体
	showstringspaces=false,%不显示代码字符串中间的空格标记
	stringstyle=\ttfamily, % 代码字符串的特殊格式
	keepspaces=true, %
	breakindent=22pt, %
	numbers=left,%左侧显示行号 往左靠,还可以为right,或none,即不加行号
	stepnumber=1,%若设置为2,则显示行号为1,3,5,即stepnumber为公差,默认stepnumber=1
	%numberstyle=\tiny, %行号字体用小号
	numberstyle={\color[RGB]{0,192,192}\tiny} ,%设置行号的大小,大小有tiny,scriptsize,footnotesize,small,normalsize,large等
	numbersep=8pt,  %设置行号与代码的距离,默认是5pt
	basicstyle=\footnotesize, % 这句设置代码的大小
	showspaces=false, %
	flexiblecolumns=true, %
	breaklines=true, %对过长的代码自动换行
	breakautoindent=true,%
	breakindent=4em, %
	escapebegin=\begin{CJK*},escapeend=\end{CJK*},
	aboveskip=1em, %代码块边框
	tabsize=2,
	showstringspaces=false, %不显示字符串中的空格
	backgroundcolor=\color[RGB]{245,245,244},   %代码背景色
	%backgroundcolor=\color[rgb]{0.91,0.91,0.91}    %添加背景色
	escapeinside=``,  %在``里显示中文
	%% added by http://bbs.ctex.org/viewthread.php?tid=53451
	fontadjust,
	captionpos=t,
	framextopmargin=2pt,framexbottommargin=2pt,abovecaptionskip=-3pt,belowcaptionskip=3pt,
	xleftmargin=1em,xrightmargin=1em, % 设定listing左右的空白
	texcl=true,
	% 设定中文冲突,断行,列模式,数学环境输入,listing数字的样式
	extendedchars=false,columns=flexible,mathescape=true
	% numbersep=-1em
}
\begin{document}
 	\maketitle
\begin{problem}
	
	假设样本 $\left(x_1,...,x_p\right)\sim N\left(0,I_{p}\right)$,其中$p\in\{10,20\}$. 建立线性回归模型$y=\beta_0+\beta_1x_1+\beta_2x_2+\beta_3x_3+\varepsilon$, 其中回归系数 $\beta_0,\beta_1,\beta_2,\beta_3 \sim Unif(1,2)$为随机数, 误差项满足正态性假设, 即$\varepsilon \sim N(0,1)$.
	
	请重复下述过程$n$次,记录正确, 多选, 少选, 错选的次数, 其中$n\in\{200,500,1000\}$.
	\begin{enumerate}
			\item {\tt}生成$\beta_0,\beta_1,\beta_2,\beta_3$;
			\item {\tt}生成$\varepsilon$和$X$数据;
			\item {\tt}生成$Y$数据;
			\item {\tt}分别利用$AIC$准则, $BIC$准则, $R_{adj}$, $C_p$统计量寻找最优子集.
	\end{enumerate}
\end{problem}
\begin{solution}  
		{\setmainfont{Courier New Bold}  
\begin{lstlisting}[language=R]
		n <- 200
		p <- 10
		correct = 0
		add = 0
		less = 0
		error = 0
		for (i in 1:500)
		{
			xmean <- matrix(c(runif(p, 0, 0)), ncol = 1)
			xsigma <- diag(p)
			library(MASS)
			x <- mvrnorm(n,xmean,xsigma)
			beta <- matrix(runif(3, 1, 2))
			beta0 <- matrix(rep(runif(1, 1, 2), n))
			e <- matrix(rnorm(n, 0, 1))
			y <- x[,c(1:3)]%*%beta + beta0 + e
			dx <- data.frame(x)
			library(leaps)
			sub.fit <- regsubsets(y~.,data = dx)
			best.summary <- summary(sub.fit)
			k <- which.max(best.summary$adjr2) #调整R方
			k <- which.min(best.summary$bic) #BIC
			k <- which.min(best.summary$cp) #cp
			best.summary$aic = best.summary$bic - p * log(n)+2 * p #AIC
			k <- which.min(best.summary$aic)
			jg <- coef(sub.fit,k)
			xz <- matrix(rownames(data.frame(jg)),ncol = 1)
			if('X1'%in%xz & 'X2'%in%xz & 'X3'%in%xz )
			{
				if(length(xz) == 4)
					correct = correct + 1
				else
					add = add + 1
			}
			else if('X4'%in%xz | 'X5'%in%xz | 'X6'%in%xz | 'X7'%in%xz | 'X8'%in%xz | 'X9'%in%xz |'X10'%in%xz)
				error = error + 1
			else 
				less = less + 1
		}
\end{lstlisting}
\end{solution}
\end{document}
