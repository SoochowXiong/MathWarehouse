\documentclass[12pt]{article}
\usepackage[a4paper,top=2.54cm,bottom=2.54cm,left=2.8cm,right=2.8cm]{geometry}
\usepackage{enumitem}
\usepackage{tikz}
\usepackage{multirow}
\usepackage{array}
\usepackage{titlesec}
\usepackage{fancyhdr}
\usepackage{lastpage}
\usepackage{xeCJK}
\usepackage{amsmath,amsfonts,amssymb,amscd,amsthm,mathrsfs}
\usepackage{tocloft}      % 用于自定义目录格式
\usepackage{hyperref}     % 用于生成超链接
\usepackage{fancyhdr}
\usepackage{lastpage}

\setlength{\parindent}{2em}
\newtheorem{theorem}{Theorem}
\newtheorem{definition}{Definition}
\newtheorem*{problem}{\itshape Problem}
\newtheorem*{Solution}{\itshape Solution}
\renewenvironment{Solution}[1][\itshape Solution.]{%
  \par\noindent\textbf{#1}\quad\CJKfamily{SimSun}%
}{\par}

\makeatletter
\renewenvironment{proof}[1][\proofname]{\par
  \pushQED{\qed}%
  \normalfont \topsep6\p@\@plus6\p@\relax
  \trivlist
  \item[\hskip\labelsep
        \bfseries\itshape
    #1\@addpunct{.}]\ignorespaces
}{%
  \popQED\endtrivlist\@endpefalse
}
\makeatother
% 设置 hyperref 的选项
\hypersetup{
    % colorlinks=true,    % 设置超链接为彩色
    % linkcolor=blue,     % 设置目录项超链接的颜色
    urlcolor=blue,      % 设置 URL 的颜色
    pdfborder={0 0 0}   % 去掉超链接的边框
}

\pagestyle{fancy}
\fancyhf{}
\rhead{20234213001 熊雄}
\lhead{\today}
\cfoot{第 \thepage 页~~共 \pageref{LastPage} 页}
\renewcommand{\headrulewidth}{0pt}
\renewcommand{\labelenumi}{\arabic{enumi}.}
\renewcommand{\labelenumii}{(\arabic{enumii})}
\renewcommand{\today}{\number\year 年 \number\month 月 \number\day 日}

\fancypagestyle{firstpage}{
  \fancyhf{} % 清空默认样式
  \cfoot{第 \thepage 页~~共 \pageref{LastPage} 页} % 设置页脚
  \renewcommand{\headrulewidth}{0pt} % 取消页眉的横线
}
\renewcommand{\headrulewidth}{0.4pt} % 设置页眉的横线宽度

\linespread{1.6}										% 行距
\expandafter\def\expandafter\normalsize\expandafter{
	\setlength\abovedisplayskip{4pt}					% 公式前行距
	\setlength\belowdisplayskip{4pt}           % 公式后行距
}
\setlist{
	topsep    = 0pt,
	parsep    = 0pt,
	itemsep   = 1pt,
	partopsep = 3pt
}

    	\titleformat{\section}[hang]{\bfseries \large}{\thesection.}{0.5em}{}
    	\titlespacing{\section}{0em}{2ex plus 1ex minus .2ex}{1.5ex plus .2ex}
    
    	% subsection
    	\titleformat{\subsection}[hang]{\bfseries}{\thesubsection.}{0.5em}{}
    	\titlespacing{\subsection}{0em}{1.5ex plus 1ex minus .2ex}{0.5ex plus .2ex}

\newcommand{\E}[1]{\mathbb{E}\left[#1\right]}
\newcommand{\dd}[1]{\,\mathrm{d}#1}

\begin{document}

\vspace{1em}
\begin{center}
\textbf{\LARGE 《金融随机分析》补充作业}\par
\vspace{6pt}
\end{center}
\thispagestyle{firstpage}
\renewcommand{\arraystretch}{1.4} % 设置行间距
\begin{center}
  \large
  \begin{tabular}{c c c}
    \multicolumn{3}{c}{2023-2024 学年第二学期 \hspace{3em} 年级: \underline{\hspace{0.5cm}\textbf{2023级金融工程}\hspace{0.5cm}}} \\
    姓名: \underline{\hspace{1cm}\textbf{熊雄}\hspace{1cm}} & 学号: \underline{\hspace{0.3cm}\textbf{20234213001}\hspace{0.3cm}} & 成绩: \underline{\hspace{3.2cm}} \\
\end{tabular}
\end{center}

\hrule height 1pt
%\noindent
%\rule{\textwidth}{1pt}
\begin{problem}
  设$Z_t,\ t\ge 0$为连续时间不可约马尔科夫链, 状态空间为$E=\{e_1,e_2,\cdots, e_N\}$, 其中
  \begin{eqnarray*}
    {e}_i = (0, \ldots, 0, \underset{\text{第}i \text{个}}{1}, 0, \ldots, 0)^T \in \mathbb{R}^N.
  \end{eqnarray*}
  $Z(t)$的转移强度矩阵为
  \begin{eqnarray*}
    A = \begin{pmatrix}
      a_{11} & a_{12} & \cdots & a_{1N} \\
      a_{21} & a_{22} & \cdots & a_{2N} \\
      \vdots & \vdots & \ddots & \vdots \\
      a_{N1} & a_{N2} & \cdots & a_{NN}
      \end{pmatrix}.
  \end{eqnarray*}
  定义 $\vec{\theta} = (\theta_1, \theta_2, \ldots, \theta_N)$, $\theta_i > 0, \ i = 1, \ldots, N$.
  \begin{enumerate}
    \item 证明:\begin{eqnarray*}
      \mathbb{E}\left[ e^{i \int_0^T \left \langle\vec{\theta},Z_t  \right \rangle dt} \right] =\left \langle e^{\left(A^T + i \operatorname{diag} \vec{\theta}\right) T} Z_0, \mathbf{1} \right \rangle,
    \end{eqnarray*}
    其中 $\mathbf{1} = (1, 1, \ldots, 1)^T$.
    \item $\lambda(t)$ 是 Cox 过程 $N_t$ 的强度过程, $\lambda(t) = \left \langle \vec{\theta}, Z_t \right \rangle$. 令 $T_1$ 是过程 $N_t$ 的首次跳时刻, 求 $P(T_1 > t)$.
    \item 计算 $ \mathbb{E}\left[ Z_T e^{i \int_0^T \left \langle\vec{\theta},Z_t  \right \rangle dt}  \right]$.
  \end{enumerate}
\end{problem}

\hrule height 1pt
\vspace{0.8cm}
\begin{enumerate}
  \item \begin{proof}
首先我们证明
\begin{eqnarray}
  Z_t = Z_0 + \int_0^t A^T Z_udu + \mathcal{M}_t, \label{Z_t}
\end{eqnarray}
其中$\mathcal{M} = \left\{\mathcal{M}_t , t\ge 0\right\}$是$(\mathcal{F}_t,P)$-鞅. 
考虑矩阵指数$e^{A^T(t-s)}$, 由$Z_t$的Markov性知
\begin{eqnarray}
  \mathbb{E}(Z_t|Z_s) = e^{A^T(t-s)}Z_s, \quad \forall t\ge s.
\end{eqnarray}
于是对于$\forall t\ge s$, 设$I$是$N\times N$的单位矩阵, 我们有
\begin{eqnarray}
  \begin{aligned}
    \mathbb{E}\left(\mathcal{M}_t-\mathcal{M}_s \mid \mathcal{F}_s\right) & = \mathbb{E}\left(Z_t-Z_s \mid \mathcal{F}_s\right)- \mathbb{E}\left(\int_s^t A^T Z_u d u \big| \mathcal{F}_s\right) \\
    & =e^{A^T(t-s)} Z_s-Z_s-\int_s^t A e^{A(u-s)} Z_s d u \\
    & =\left(e^{A^T(t-s)}-I-\int_s^t A^T e^{A(u-s)} d u\right) Z_s \\
    & =\left(e^{A^T(t-s)}-I-\left[e^{A^T(u-s)}\right]_s^t\right) Z_s \\ 
    & =0,
    \end{aligned}
\end{eqnarray}
因此\eqref{Z_t}成立. 

下面考虑过程
\begin{eqnarray}
  X_t :=  e^{i  \int_0^t \left \langle\vec{\theta},Z_t  \right \rangle dt}Z_t. 
\end{eqnarray}
我们通过计算可以得到
\begin{eqnarray*}
    d X_t & =& e^{i\int_0^t\left \langle\vec{\theta}, Z_u\right \rangle d u}\cdot d Z_t+e^{ i\int_0^t\left \langle\vec{\theta}, Z_u\right \rangle d u} \cdot i\left \langle\vec{\theta}, Z_t\right \rangle Z_t d t \\
    & =&\left(A^T +i \operatorname{diag} \vec{\theta}\right) X_u d u+e^{ i\int_0^t\left \langle\vec{\theta}, Z_u\right \rangle d u} d \mathcal{M}_t.
\end{eqnarray*}
因此
\begin{eqnarray}
  X_t = Z_0 + \int_0 ^ t \left(A^T +i \operatorname{diag} \vec{\theta}\right) X_u d u + \int_0^t e^{ i\int_0^u\left \langle\vec{\theta}, Z_u\right \rangle d u} d \mathcal{M}_u,\label{X_t}
\end{eqnarray}
其中积分$ \int_0^t e^{ i\int_0^u\left \langle\vec{\theta}, Z_u\right \rangle d u} d \mathcal{M}_u$是一个鞅, 因此对\eqref{X_t}取期望有
\begin{eqnarray*}
  \mathbb{E}\left[X_t\right] = Z_0 + \int_0 ^ t \left(A^T +i \operatorname{diag} \vec{\theta}\right)  \mathbb{E}\left[X_u\right]  d u.
\end{eqnarray*}
这是一个积分方程, 易解得
\begin{eqnarray}
  \mathbb{E}\left[X_t\right] =  e^{\left(A^T + i \operatorname{diag} \vec{\theta}\right) t}\cdot Z_0. \label{ode Solution}
\end{eqnarray}
$ \mathbb{E}\left[X_t\right]$是$\mathbb{R}^N$上的期望过程, 特征函数
\begin{eqnarray*}
  \mathbb{E}\left[ e^{i \int_0^T \left \langle\vec{\theta},Z_t  \right \rangle dt} \right]
\end{eqnarray*}
可以通过对其分量求和可以得到. 因此
\begin{eqnarray}
  \mathbb{E}\left[ e^{i \int_0^T \left \langle\vec{\theta},Z_t  \right \rangle dt} \right] = \mathbb{E}\left[ \left \langle e^{i \int_0^T \left \langle\vec{\theta},Z_u  \right \rangle du} Z_T, \mathbf{1} \right \rangle \right] =\left \langle e^{\left(A^T + i \operatorname{diag} \vec{\theta}\right) T} Z_0, \mathbf{1} \right \rangle.
\end{eqnarray}
证毕.
  \end{proof}
  \item \begin{Solution}
    Cox 过程是一种非齐次泊松过程, 其强度过程为 $\lambda(t)$.对于$N_t$过程的首次跳时刻 $T_1$, 其分布函数满足
    \begin{eqnarray}
      P(T_1 > t) = \mathbb{E}\left[e^{-\int_0^t \lambda(s) ds}\right],
    \end{eqnarray}
    这里$\lambda(t) = \left\langle \vec{\theta}, Z_t \right\rangle$, 所以我们有
    \begin{eqnarray}
      P(T_1 > t) = \mathbb{E}\left[e^{-\int_0^t \left\langle \vec{\theta}, Z_s \right\rangle ds}\right].
    \end{eqnarray}
利用第1问中我们得到的结果, 我们有
    \begin{eqnarray}
      \mathbb{E}\left[e^{-\int_0^t \left\langle \vec{\theta}, Z_s \right\rangle ds}\right] = \left\langle e^{\left(A^T - \operatorname{diag} \vec{\theta}\right) t} Z_0, \mathbf{1}\right\rangle.
    \end{eqnarray}
因此可以得到
    \begin{eqnarray}
      P(T_1 > t) = \left\langle e^{\left(A^T - \operatorname{diag} \vec{\theta}\right) t} Z_0, \mathbf{1} \right\rangle.
    \end{eqnarray}    
  \end{Solution}
  \item \begin{Solution}
    考虑过程
\begin{eqnarray*}
  X_t :=  e^{i  \int_0^t \left \langle\vec{\theta},Z_t  \right \rangle dt}Z_t. 
\end{eqnarray*}
由第1问证明过程中的式\eqref{ode Solution}知
\begin{eqnarray}
  \mathbb{E}\left[X_T\right] =  e^{\left(A^T + i \operatorname{diag} \vec{\theta}\right) T}\cdot Z_0,
\end{eqnarray}
即
\begin{eqnarray}
  \mathbb{E}\left[ Z_T e^{i \int_0^T \left \langle\vec{\theta},Z_t  \right \rangle dt}  \right] = e^{\left(A^T + i \operatorname{diag} \vec{\theta}\right) T}\cdot Z_0.
\end{eqnarray}
  \end{Solution}
\end{enumerate}
\end{document}
